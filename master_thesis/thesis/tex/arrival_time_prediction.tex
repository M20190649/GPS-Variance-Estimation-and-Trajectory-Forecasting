\chapter{Arrival Time Prediction}
\label{cha:arrival-time-prediction}

\todo{Arrival Time Prediction -> ...}
Trajectory Forecasting is the process of predetermining when a bus will arrive at a particular bus stop.
The existing Trajectory Forecasting is based purely on long-time historical data.
It provides a single timestamp when it predicts the bus to arrive at a certain bus stop.


\section{Methodology}

This thesis project proposes a Trajectory Forecasting model based on multiple GPs.
It provides time to arrival distributions to all the remaining bus stops, meaning that it can answer queries such as:
\begin{itemize}
    \item \textit{"What is the most likely time of arrival at bus stop $p$?"}
    \item \textit{"What is the probability of the bus arriving at the earliest/latest possible time?"}
    \item \textit{Which stop impacts the arrival time of the bus the most?}
\end{itemize}

The approach has three steps:
\begin{enumerate}
    \item \textit{Stop Model and Compression:}
    Stops will be modelled as both bus stops and red lights, i.e., common stops during a bus journey.
    GPS positions of stopped buses need to be compressed in order to support more robust GPs \todo{source: Gaussian Process Based Motion Pattern Recognition
    with Sequential Local Models}.
    The result of this step is a timeline for each bus journey with marked common stops and compressed GPS positions during stops.

    \item \textit{Segment Stops (With Model Overlap):}
    The next step is to process the timeline and segment it between stops.
    The segments need to be checked for self-overlaps, in order to avoid the problems described in Section \todo{ref{sec:problems-trajectory-forecasting}}.
    Each segment will have a small model overlap, which means that neighbouring segments will share a few data points.
    This improves the robustness of the model.

    \item \textit{GP Trajectory Models:}
    The final step is to fit multiple GPs to each trajectory segment.
    The first GP fits $x$, $y$ (longitude, latitude) coordinates to a time $\tau$:
    \begin{equation}
       f_1: x, y \longmapsto \tau
    \end{equation}
    
    $\tau$ is the relative time of the journey, denoting the progress of the bus in completing the journey.
    The second and third GP fits $\tau$ to a new $x$ and $y$, respectively:
    \begin{equation}
        f_2: \tau \longmapsto x
    \end{equation}
    \begin{equation}
        f_3: \tau \longmapsto y
    \end{equation}
\end{enumerate}
