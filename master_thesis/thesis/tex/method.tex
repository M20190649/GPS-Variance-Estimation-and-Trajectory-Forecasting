\chapter{Methodology}
\label{cha:methodology}

This chapter describes the methodology applied to solve two of the three higher-level problems presented in this thesis project.

\section{GPS Variation Estimation}
The method used in this thesis project to solve the higher-level GPS Variation Estimation problem is based on combined Gaussian Processes (GPs).
The method is described by 5 steps:

\begin{enumerate}
    \item \textit{Stop Compression:}
    Each stop present in the journey causes an imbalance of points being clustered around each stop.
    The longer the stop is, the more points are present.
    This results in problems with the GP, as certain coordinates would be denser than others.
    Positions sent while a bus has stopped are averaged into a single coordinate.
    The time of the stop is also adjusted.
    The result is journey consisting of a sequence of coordinates where stops are invisible both time-wise and position-wise.

    \item \textit{Segment Journeys:}
    Each bus journey in the dataset is fitted with its own GPs.
    Journeys are analysed in order to detect self-overlapping, which would cause problems for the fitted GPs.
    Journeys with self-overlap are split at the overlapping points into two segments.
    The process recursively analyses each resulting segment and proceeds with splitting the segments as long as self-overlapping is present.
    If a journey has multiple segments, each segment is fitted with its own GPs.

    The first GP fits $x$, $y$ (latitude, longitude) coordinates to a new time parameter $\tau$:
    \begin{equation} \label{eq:gps-var-f1}
       f_1: x, y \longmapsto \tau
    \end{equation}
    $\tau$ is the relative time of the journey, denoting the progress of the bus in completing the journey.
    The second and third GP fits $\tau$ to a new $x$ and $y$, respectively:
    \begin{equation} \label{eq:gps-var-f2}
        f_2: \tau \longmapsto x
    \end{equation}
    \begin{equation} \label{eq:gps-var-f3}
        f_3: \tau \longmapsto y
    \end{equation}

    \item \textit{Point-Wise Combining of GP:}
    The GPs $f_2$ (Eq. \ref{eq:gps-var-f2}) and $f_3$ (Eq. \ref{eq:gps-var-f3}) each describe the variance ($\sigma_2^2$ and $\sigma_3^2$, respectively) at each point in the fitted segment or journey.
    The GPs describing the variances are point-wise combined using Equation \ref{eq:mean-point-wise-combined} and \ref{eq:var-point-wise-combined}, creating the GPs $f_4$ (Eq. \ref{eq:gps-var-f4}) and $f_5$ (Eq. \ref{eq:gps-var-f5}), respectively.
    
    \begin{equation} \label{eq:mean-point-wise-combined}
        \mu(x^*) = \frac{\sum_{j=1}^{J} N_j\mu_j(x^*)}{\sum_{j=1}^{J} N_j},
    \end{equation}

    \begin{equation} \label{eq:var-point-wise-combined}
        \sigma^2(x^*) = \frac{\sum_{j=1}^{J} N_j(\sigma_j^2(x^*) + \mu_j^2(x^*))}{\sum_{j=1}^{J} N_j} - \mu(x^*)^2,
    \end{equation}

    \begin{equation} \label{eq:gps-var-f4}
        f_4: \tau \longmapsto \sigma_4^2
    \end{equation}
    \begin{equation} \label{eq:gps-var-f5}
        f_5: \tau \longmapsto \sigma_5^2
    \end{equation}
    where $\sigma_4^2$ is the variation in latitude and $\sigma_5^2$ the variation in longitude.

    \item \textit{Visualisation:}
    The resulting $f_4$ and $f_5$ GPs are then visualised. \todo{hur?}

    \item \textit{Comparison:}
    The visualised GPs are compared and different routes are compared at the same spatial area.
    The comparison analyses if there is a periodicity to the variation or if the variation is purely based on spatial locality.
\end{enumerate}  

\section{Trajectory Forecasting}
This thesis project proposes a Trajectory Forecasting model based on multiple GPs.
It provides time to arrival distributions to all the remaining bus stops, meaning that it can answers queries such as:
\textit{"What is the most likely time of arrival at station $p$?"}, \textit{"What is the probability of the bus arriving at the earliest/latest possible time?"}, and \textit{Which stop impacts the arrival time of the bus the most?}.

The approach has three steps:
\begin{enumerate}
    \item \textit{Stop Model and Compression:}
    Stops will be modelled as both bus stops and red lights, i.e., common stops during a bus journey.
    GPS positions of stopped buses needs to be compressed in order to support more robust GPs \todo{source: Gaussian Process Based Motion Pattern Recognition
    with Sequential Local Models}.
    The result of this step is a timeline for each bus journey with marked common stops and compressed GPS positions during stops.

    \item \textit{Segment Stops (With Model Overlap):}
    The next step is to process the timeline and segment it between stops.
    The segments need to be checked for self-overlaps, in order to avoid the problems described in Section \ref{sec:problems-trajectory-forecasting}.
    Each segment will have a small model overlap, which means that neighbouring segments will share a few data points.
    This improves the robustness of the model.

    \item \textit{GP Trajectory Models:}
    The final step is to fit multiple GPs to each trajectory segment.
    The first GP fits $x$, $y$ (latitude, longitude) coordinates to a time $\tau$:
    \begin{equation}
       f_1: x, y \longmapsto \tau
    \end{equation}
    
    $\tau$ is the relative time of the journey, denoting the progress of the bus in completing the journey.
    The second and third GP fits $\tau$ to a new $x$ and $y$, respectively:
    \begin{equation}
        f_2: \tau \longmapsto x
    \end{equation}
    \begin{equation}
        f_3: \tau \longmapsto y
    \end{equation}
\end{enumerate}

