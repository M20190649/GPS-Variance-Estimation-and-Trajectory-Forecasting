\chapter{Introduction}
\label{cha:introduction}

%\section{Motivation}
%\label{sec:motivation}
Today, there is an increasing number of services built upon the use of spatio-temporal models.
For example, public transportation companies use such spatio-temporal models for route planning and arrival time prediction.
Taxi companies use spatio-temporal data to track vehicles and, for example, assign the closest available vehicle to a new customer.
Spatio-temporal datasets can also be used for city planning and analysing effects of rapid urbanisation \cite{Deng2009, Liu2005}.
Service vehicles are today equipped with GPS sensors which collect behavioural data.
The availability and usability of spatio-temporal datasets will only increase with the development of new technologies.
For example, the field of autonomous vehicles utilises spatio-temporal dataset for various applications \cite{Dickmanns1990, Peng2018}.

The purpose of this thesis project is to analyse a novel spatio-temporal dataset collected by Östgötatrafiken AB.
The goal is to understand the dataset, find out if there are inherent problems in the dataset, and create models to answer questions about reality, based on the dataset.
For example, models can be used to gain further knowledge about the flow of traffic.
New or improved services can be provided by creating models of specific aspects captured by the dataset.
The arrival time prediction or vehicle position estimation for service vehicles can be improved.
The provided documentation for the dataset is minimal, which makes the insights from the analysis even more valuable.

\section{Aim}
\label{sec:aim}
The aim of this thesis project is to explore a novel dataset and formulate problems and solutions which can be interesting not only to the dataset provider but also in a broader context, e.g., for services relying on GPS positioning.
Currently, Östgötatrafiken AB is using models based on a similar dataset to provide time forecasts on the public transportation available in Östergötland county.
The aim is not to answer all questions the dataset could provide (partial) answers to, but rather outline a few interesting questions and problems.
Some of the problems are implemented in this thesis project and their results presented.

\section{Research Questions}
\label{sec:research-questions}
The specific research questions treated in this thesis project are divided into three groups: \textit{Metadata Questions}, \textit{High-Level Problem Questions}, and \textit{Problem Specific Questions}.
The \textit{Metadata Questions} investigate and explore the provided dataset in depth, using Exploratory Data Analysis (EDA).
They cover issues such as pre-processing, data noise and outliers.
These questions can be seen as a pre-study before data analysis and machine learning techniques are applied.
The insights from these questions are the basis for the \textit{High-Level Problem Questions}.
The \textit{High-Level Problem Questions} formulate general problems which span individual applications by using the information available both internally in the dataset and externally from sources outside the provided dataset.
High-level problems are problems which are not inherent in the dataset but rather problems which can be solved by using the information available in the dataset.
The \textit{Problem Specific Questions} are related to a specific problem described by the \textit{High-Level Problem Questions}.

\begin{description}
  \item \textbf{Metadata Questions:}
  \begin{enumerate}
    \item \textit{What kind of information is available in the dataset provided by Östgötatrafiken AB?} \newline
    What features exist in the dataset, such as different event types, event parameters, and event structures?
    What do these features mean and how do they relate to each other?
    Are some features more important than others?
    Are there any features missing from the given dataset?
    \item \textit{What general pre-processing can be done in order to use the dataset to solve high-level problems?} \newline
    The pre-processing focuses on solving problems inherent in the dataset, such as processing events and extracting relevant information from them.
    What are some pre-processing methods that can be applied to this dataset?
    How can the raw data in the dataset be transformed into useful features?
    How does the ordering of the raw data affect the solutions for high-level problems?
    How can the raw data be transformed into trajectories?
    How can information from different event types be added to the trajectories?
    \item \textit{How can noisy measurements be detected?} \newline
    Noisy measurements can affect the solutions for high-level problems negatively.
    Various anomaly-detection algorithms can be applied to find such cases, but they can also be solved using dataset-tailored algorithms.
    Which manual inspection methods can be used to detect noisy measurements?
    Which general anomaly detection algorithms can be used?
    How can a dataset-tailored algorithm be implemented?
    How does the general anomaly detection algorithm differ from the dataset-tailored algorithm?
    \item \textit{How is the provided dataset related to external data sources and how can they be combined?} \newline
    There are external data sources available which can complement the data in the provided dataset, such as Google Roads API\footnote{https://developers.google.com/maps/documentation/roads/intro}, the website of Östgötatrafiken\footnote{https://www.ostgotatrafiken.se/} and the Traffic Lab API\footnote{https://www.trafiklab.se/}.
    How compatible is the data from the external sources with the data in the given dataset?
    Does the external sources contain any of the features that are deemed missing from the given dataset (in \textit{Metadata Question} 1)? 
    How can the data from the external sources be used?
  \end{enumerate}

  \item \textbf{High-Level Problem Questions:}
  \begin{enumerate}
    \item \textit{What interesting high-level problems can be investigated and solved based on the available data?} \newline
    The high-level problems can utilise both the data available in the provided dataset and any complementary external data.
    What problems span over the areas of GPS Positioning and Trajectory Forecasting?
    What are the core problems for each area?
    How can the existing system that generated the dataset be improved?
    \item \textit{How can these problems be solved?} \newline
    The solutions tackle the core of each problem.
    Each solution explicitly states if there is a baseline available for comparison or if one can be easily created.
    How can problems with GPS Positioning and Trajectory Forecasting be solved?
    \item \textit{How do solutions to these problems benefit society and the industry?} \newline
    The solutions are analysed in a broader context, e.g., from a societal or ethical point of view.
    What value do the solutions offer to the industry?
    How is consumer privacy affected by the solutions?
  \end{enumerate}

  \item \textbf{Problem Specific Questions}
  \begin{enumerate}
    \item \textit{How can the spatio-temporal varying GPS variance be estimated from sets of observed trajectories over extended periods of time?}
    Can recent Gaussian Processes Regression (GPR) based trajectory modelling approaches be used to solve this problem?
    How can the GPR-based approach scale with multiple trajectory models?
    How can potential periodicity of the data be handled in the approach?
    How can the model be trained on extended periods of time?
    \item \textit{How can the approach to estimate the spatio-temporal varying GPS variance be evaluated?} \newline
    What assumptions are made regarding the inherent noise of the data?
    How can kernels be evaluated if a GPR-based approach is applied?
    How can trajectories be evaluated?
    Which evaluation criteria can be applied to evaluate the approach?

    \item \textit{How can Trajectory Forecasting be implemented from sets of observed trajectories?} \newline
    Can GPR-based trajectory modelling approaches be used to solve this problem?
    Which modifications need to be done to the approach in order to produce trajectory forecasts?
    How can features from observed trajectories over extended periods of time be used to improve the forecasts?

    \item \textit{How can the Trajectory Forecasting model be evaluated?} \newline
    Can the existing forecasts from Östgötatrafiken AB be used to create an evaluation baseline?
    How can new forecasts be compared with the baseline?
    Which insights can be made from the information available in the output of the Trajectory Forecasting model?
    How can the model respond to immediate real-time changes?
    How can the performance of a single forecast be evaluated?
    How can the overall performance of the model be evaluated?
    Which evaluation criteria can be applied to evaluate the forecasts?
    Can the new model be combined with the existing model to produce more precise forecasts?
    Are there any benefits to returning a probability distribution over arrival times compared to returning the expected arrival time?
  \end{enumerate}
\end{description}

\section{Delimitations and Limitations}
\label{sec:delimitations}
The dataset is provided by Östgötatrafiken AB and is not available for public use.
This thesis project only focuses on the data from buses; data from public transportation trains are ignored.
To support manual inspection of the data in the dataset, it is filtered to only contain data from a certain geographical area.
The data gathering process cannot be changed or affected during the span of this thesis project; the dataset is thus as-is, which means that information missing in the dataset cannot be retrieved from other sources.
Bugs in the data gathering process cannot be solved either.

The models used in the thesis project are trained and evaluated on a MacBook Pro (early 2015), with a 2.7 GHz Intel Core i5 processor and 8GB 1867 MHz DDR3 memory.
The setup is by no means optimal, as the MacBook Pro used does not have a GPU.
The TensorFlow library is thus not fully utilised, as there is no GPU acceleration.

\section{Structure of the Report}
Chapter \ref{cha:background} covers the background concepts used in the report.
Relevant work is also described, alongside the methodology employed in the thesis project.
Chapter \ref{cha:data} analyses the data in depth.
The data gathering process is explained together with the structure and characteristics of the dataset.
In the dataset pre-study, problems inherent in the dataset are explained and solutions to the problems are either outlined or implemented.
Interesting problems from real-world scenarios are visualised and used to explain various pre-processing steps.
The data chapter acts as a stepping stone to the high-level problems described in Chapters \ref{cha:bus-stop-detection}-\ref{cha:arrival-time-prediction}.
Chapter \ref{cha:bus-stop-detection} covers the problem of poorly detected bus stops, which causes a large proportion of the problems in the dataset.
The chapter explains the problem and proposes a solution to it.
Chapter \ref{cha:GPS-variation-estimation} focuses on the positional data in the dataset.
It attempts to estimate the GPS variation of journeys for a bus line.
If the proposed methodology yields good results, the overall precision of positional data can be improved for the dataset.
Chapter \ref{cha:arrival-time-prediction} proposes a method to predict the arrival time of buses, using processed data from the spatio-temporal dataset.
The chapter covers the description of the problem, the methodology, and the results of the implemented solution.