\chapter{Introduction}
\label{cha:introduction}

\section{Motivation}
\label{sec:motivation}

\section{Aim}
\label{sec:aim}
The aim of this thesis project is to explore a novel dataset and formulate problems and solution suggestions which could be interesting to not only the dataset provider but also in a broader context.
The dataset is currently used to provide time forecasts on the public transportation available in Östergötland county.



\section{Research Questions}
\label{sec:research-questions}
The specific research questions treated in this thesis project is presented in this section.
They are divided into three groups: \textit{Metadata Questions}, \textit{Higher-Level Problem Questions}, and \textit{Problem Specific Questions}.
The \textit{Metadata Questions} investigate and explore the provided dataset in depth.
They cover issues such as pre-processing, data noise and outliers.
The questions can be seen as a pre-study to Machine Learning.
The insights from these questions are the basis for the \textit{Higher-Level Problem Questions}.
The \textit{Higher-Level Problem Questions} formulate various high-level problems by using the information available both internally in the dataset and externally from sources outside the provided dataset.
High-level problems are problems which are not inherent in the dataset but rather problems which can be solved by using the information available in the dataset.
They also aim to suggest solutions for these formulated problems.
The \textit{Problem Specific Questions} are questions related to a specific problem and solution described by the \textit{Higher-Level Problem Questions}.

\begin{description}
  \item \textbf{Metadata Questions:}
  \begin{enumerate}
    \item \textit{What kind of information is available in the dataset provided by Östgötatrafiken AB?} \newline
    This question explores what kind of data the dataset contains.
    The dataset is a novel dataset which has not been worked on before.
    The provided documentation is minimal, which makes this question non-trivial and the insights from the question even more valuable.
    It analyses the features of the dataset, e.g. different event types, event parameters, and event structure.
    \item \textit{What pre-processing needs to be done in order to solve higher-level problems?} \newline
    The higher-level problems are here defined as problems which are not inherent inside the dataset but rather problems which can be solved by using the dataset.
    The pre-processing focuses instead of solving problems inherent in the dataset, such as processing events and extracting relevant information from them or building a knowledge base by looking at the order of events.
    \item \textit{How can noisy measurements be detected?} \newline
    Noisy measurements can affect the solutions for higher-level problems negatively.
    Various anomaly detection algorithms can be applied to detect such cases, but they could also be solved using dataset-tailored algorithms.
    \item \textit{How is the provided dataset related to external data sources and how can they be combined?} \newline
    There are external data sources available which could complement the data in the provided dataset.
    This question explores if these sources could be combined in order to solve problems on a higher level. 
  \end{enumerate}

  \item \textbf{Higher-Level Problem Questions:}
  \begin{enumerate}
    \item \textit{What are some of the higher-level problems which can be formulated using the available data?} \newline
    These problems can utilise both the data available in the provided dataset and any complementary external data.
    The answer to this question will not be a complete list of all possible problems, but rather a short list of a few interesting examples.
    Each formulated problem will have its core problem explained.
    \todo{Fråga: Ska jag lista de problem jag har redan här? Annars blir det svårt att fortsätta med Problem Specific Question.}
    \item \textit{What are some of the solutions to the formulated problems?} \newline
    The list of solutions for each formulated problem will not be a complete list of all possible solutions.
    The solutions will be tackling the core of each formulated problem.
    Each solution will explicitly state if there is a baseline available for comparison or if one could be easily created.
    \item \textit{Who could benefit from the solutions?} \newline
    This question analyses the solutions in a broader context, e.g. from a societal or ethical point of view.
    For example, a solution could yield great results for the industry at the cost of consumer privacy.
  \end{enumerate}

  \item \textbf{Problem Specific Questions}
  \begin{enumerate}
    \item \textit{How can GPS variance estimation be solved with combined Gaussian Processes Regression using a local trajectory model?} \newline
    \todo{This question is very explicit and stems from the results from asking the questions in the two previous groups. Is this okay, or do I need to add more information after each question in the previous groups? In one way this question kind of pops up from nowhere.}
    \item \textit{How can the GPS variance estimation model be evaluated?} \newline
    The method employed makes certain assumptions regarding the inherent noise of the data.
    The kernels applied to the model need to evaluated.
    The local trajectory model shall be compared with a global trajectory model.
    \item \textit{How can Trajectory Forecasting be realised with Gaussian Processes Regression using a local trajectory model?} \newline
    \item \textit{How can the Trajectory Forecasting model be evaluated in the context of information gain?} \newline
    The Trajectory Forecasting model could, for example, be evaluated by comparing the new forecasts with the existing forecasts from the baseline created by the internal system of Östgötatrafiken AB.
    This evaluation leads to information regarding which model to use to get more precise forecasts.
    The model could also be evaluated by looking at what kind of information and insights are available from the output of the model.
    A model which returns a probability distribution of arrival times, which are updated in real-time, should be evaluated in a broader context than precision of a single forecast.
    \todo{Det här kanske inte hör hemma här. Det är väldigt klumpigt skrivet iaf.}
  \end{enumerate}
\end{description}

\section{Delimitations}
\label{sec:delimitations}
The dataset is provided by Östgötatrafiken AB and is not available for public use.
This thesis project will only focus on the bus data in the dataset, data from public transportation trains will be ignored.
In order to support manual inspection of the data in the dataset, the data is filtered to only contain data from a certain geographical area.
